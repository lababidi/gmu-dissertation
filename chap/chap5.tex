

\chapter{Summary and Outlook}

Thus far, we have delved into the physics of heterostructures of superconductors and topological insulators, starting with the TI's interaction with a metal and ending with a Josephson junction on the surface of the TI. 

We found that electrons traveling from the metal to the surface of the TI can have a perfect spin flip under certain conditions. 
In addition we found that there occurs a hybridization between the metal and the surface of the TI, where the spectrum near the surface of the TI resembles that of the metal when the metal is strongly in contact with the TI. 
One possibility to extend the spin-flip mechanism found would be to have two surfaces of TIs sandwiching a metal. This flat 2D quantum device could have implications in spintronics applications.

In the study of the heterostructure of a superconductor and TI, we found that there does exist a subgap mode that penetrates deep into the superconductor. The parameters that describe this mode are renormalized from the respective bulk values of the individual materials due to the interplay between the TI and superconductor. A serious possibility on continuing this focus of microscopic simulation of a S-TI heterostructure is by simulating, more realistically, a Weyl superconductor, a periodic array of S-TI heterostructures with magnetic doping on the TI segments. The Weyl superconductor is an exotic gapless superconductor\cite{meng_weyl_2012}. This prospective direction has experimental implications due to experimental realizations in magnetically doped TIs\cite{liu_magnetic_2009,chen_massive_2010,zhang_topology-driven_2013}. 

The Josephson junction on the surface of the TI gives rise to some very unique phenomena. The energy spectrum shows that when the junction's phase is $\pi$, the linear energy dispersion morphs into a flat, zero-slope dispersion as the chemical potential is tuned away from zero. This dispersion also presents a strong peak in the density of states. 
The progression from this study has a few directions. The flat band, illustrates that the quasiparticle excitations have ``low" kinetic energy and therefore secondary interactions, if they can be induced, can lead to new phases. And lastly, the ground state of the junction can be determined by comparing the free energy for different phase biases. This realistic study could precipitate further experiments to find the zero-energy Majorana mode.

